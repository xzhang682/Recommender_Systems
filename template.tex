\documentclass{article}

\usepackage{amssymb}
\usepackage{amsmath, amsthm, color}
\usepackage[colorlinks]{hyperref}
\usepackage{fullpage}
\usepackage{verbatim}

\usepackage{lineno}
%\linenumbers



\newtheorem{theorem}{Theorem}
\newtheorem{question}[theorem]{Question}
\newtheorem{corollary}[theorem]{Corollary}
\newtheorem{conjecture}[theorem]{Conjecture}
\newtheorem{lemma}{Lemma}
\newtheorem{proposition}[theorem]{Proposition}

% % the following creates defns with the usual roman font, not in italics
\newtheorem{definition}{Definition}
\newtheorem{fact}{Fact}[section]
\newtheorem{assumption}{Assumption}

% Not in italics
\theoremstyle{definition}
\newtheorem{remark}[theorem]{Remark}
\newtheorem{example}[theorem]{Example}

% WIDEBAR COMMAND
\newlength{\widebarargwidth}
\newlength{\widebarargheight}
\newlength{\widebarargdepth}
\DeclareRobustCommand{\widebar}[1]{%
  \settowidth{\widebarargwidth}{\ensuremath{#1}}%
  \settoheight{\widebarargheight}{\ensuremath{#1}}%
  \settodepth{\widebarargdepth}{\ensuremath{#1}}%
  \addtolength{\widebarargwidth}{-0.3\widebarargheight}%
  \addtolength{\widebarargwidth}{-0.3\widebarargdepth}%
  \makebox[0pt][l]{\hspace{0.3\widebarargheight}%
    \hspace{0.3\widebarargdepth}%
    \addtolength{\widebarargheight}{0.3ex}%
    \rule[\widebarargheight]{0.95\widebarargwidth}{0.1ex}}%
  {#1}}

% Notational convenience

\newcommand{\E}{\ensuremath{\mathbb{E}}}
\newcommand{\mprob}{\ensuremath{\mathbb{P}}}
\newcommand{\scriptW}{\ensuremath{\mathcal{W}}}
\newcommand{\real}{\ensuremath{\mathbb{R}}}
\newcommand{\muhat}{\ensuremath{\widehat{\mu}}}
\newcommand{\barw}{\ensuremath{\bar{w}}}
\newcommand{\scriptE}{\ensuremath{\mathcal{E}}}
\newcommand{\scriptT}{\ensuremath{\mathcal{T}}}
\newcommand{\sigmatil}{\ensuremath{\widetilde{\sigma}}}
\newcommand{\order}{\ensuremath{\mathcal{O}}}
\newcommand{\scriptA}{\ensuremath{\mathcal{A}}}
\newcommand{\mustar}{\ensuremath{\mu^*}}
\newcommand{\thetabar}{\ensuremath{\widebar{\theta}}}
\newcommand{\poly}{\ensuremath{\operatorname{poly}}}
\newcommand{\Var}{\ensuremath{\operatorname{Var}}}
\newcommand{\Otil}{\ensuremath{\widetilde{O}}}
\newcommand{\sigmahat}{\ensuremath{\widehat{\sigma}}}
\newcommand{\betastar}{\ensuremath{\beta^*}}
\newcommand{\gammastar}{\ensuremath{\gamma^*}}
\newcommand{\betahat}{\ensuremath{\widehat{\beta}}}
\newcommand{\gammahat}{\ensuremath{\widehat{\gamma}}}
\newcommand{\nuhat}{\ensuremath{\widehat{\nu}}}
\newcommand{\xihat}{\ensuremath{\widehat{\xi}}}
\newcommand{\supp}{\ensuremath{\operatorname{supp}}}
\newcommand{\Ttil}{\ensuremath{\widetilde{T}}}
\newcommand{\inprod}[2]{\ensuremath{\langle #1 , \, #2 \rangle}}
\newcommand{\Ball}{\ensuremath{\mathbb{B}}}
\newcommand{\Xtil}{\ensuremath{\widetilde{X}}}
\newcommand{\Atil}{\ensuremath{\widetilde{A}}}
\newcommand{\opnorm}[1]{\left|\!\left|\!\left|{#1}\right|\!\right|\!\right|}




\usepackage{arxiv}

\usepackage[utf8]{inputenc} % allow utf-8 input
\usepackage[T1]{fontenc}    % use 8-bit T1 fonts
\usepackage{hyperref}       % hyperlinks
\usepackage{url}            % simple URL typesetting
\usepackage{booktabs}       % professional-quality tables
\usepackage{amsfonts}       % blackboard math symbols
\usepackage{nicefrac}       % compact symbols for 1/2, etc.
\usepackage{microtype}      % microtypography
\usepackage{lipsum}
\usepackage{xcolor}
\usepackage{graphicx}
% \title{A template for the \emph{arxiv} style}


% \author{
%   David S.~Hippocampus\thanks{Use footnote for providing further
%     information about author (webpage, alternative
%     address)---\emph{not} for acknowledging funding agencies.} \\
%   Department of Computer Science\\
%   Cranberry-Lemon University\\
%   Pittsburgh, PA 15213 \\
%   \texttt{hippo@cs.cranberry-lemon.edu} \\
%   %% examples of more authors
%   \And
%  Elias D.~Striatum \\
%   Department of Electrical Engineering\\
%   Mount-Sheikh University\\
%   Santa Narimana, Levand \\
%   \texttt{stariate@ee.mount-sheikh.edu} \\
%   %% \AND
%   %% Coauthor \\
%   %% Affiliation \\
%   %% Address \\
%   %% \texttt{email} \\
%   %% \And
%   %% Coauthor \\
%   %% Affiliation \\
%   %% Address \\
%   %% \texttt{email} \\
%   %% \And
%   %% Coauthor \\
%   %% Affiliation \\
%   %% Address \\
%   %% \texttt{email} \\
% }

\begin{document}
% \maketitle

% \begin{abstract}
% \lipsum[1]
% \end{abstract}


% % keywords can be removed
% \keywords{First keyword \and Second keyword \and More}

\section{User-User Collaborative Filtering}

\subsection{Main Contents}
References: An Algorithmic Framwork for Collaborative Filtering by Herlocker, Konstan, Borchers, and Riedl (Proc. SIGIR 1999).

Let $U$ be a set of users and $I$ be a set of items. Suppose we have a collection of ratings $r_{u,i}$ for some user $u$ rating some item $i$. Note that the rating matrix $R$ is assumed to be sparse. A simple prediction score is defined as follow,
\begin{align}
    s(u,i) = \frac{\sum_{v \in U} r_{v, i} }{|U|}.
\end{align}

To weight the ratings differently, a variant is as follow
\begin{align}
    s(u,i) = \frac{\sum_{v \in U} r_{v, i} \cdot w_{u,v} }{\sum_{v\in U} w_{u, v}},
\end{align}
where $w_{u,v}$ is a similarity/weighting parameter between user $u$ and $v$. 

For the problem that users have different rating scales, letting $\bar{r}_u$ be the average score for some user $u$, we can introduce the following variant,
\begin{align}
    s(u,i) = \bar{r}_{u} + \frac{\sum_{v\in U} (r_{v, i} - \bar{r}_v)}{|U|}.
\end{align}

Combining the two variants we get 
\begin{align}
    s(u,i) = \bar{r}_{u} + \frac{\sum_{v\in U} (r_{v, i} - \bar{r}_v) \cdot w_{u, v}}{\sum_{v\in U} w_{u, v}}.
\end{align}
In particular, the weighting parameter can be defined as the Pearson correlation which is defined as 
\begin{align*}
    w_{u,v} = \frac{\sum_{i \in I} (r_{u,i} - r_u)(r_{v,i} - r_v)}{\sigma_u \cdot \sigma_v}.
\end{align*}

UUCF is a personalized recommendations/predictions.

The base assumptions of the UUCF are: Our tastes are either individually stable or move in sync with each other; Our system is scoped within a domain of agreement.

The implementation issues locates on when the number of users and items are in large scale. In particular, given $m = |U|, n = |I|$, we have correlation between two users is $O(n)$; all correlations for a user is $O(mn)$; all pairwise correlation is $O(m^2n)$; recommendations take at least $O(mn)$. 

There are lots of ways to make computation practical: more persistent neighborhoods ($m \to k$); cached or incremental correlations.

\subsection{Configuring User-User Collaborative Filtering}
\paragraph{Selecting Neighborhoods:}
We can use all the neighbors; we can threshold similarity or distance, say 0.1 fraction of neighbors; we can random pick some number of neighbors; we can pick top-N neighbors by similarity or distance; we can also cluster and users and define neighbors. 

In theory, we want more neighbors. But in practice, noise can be made from dissimilar neighbors decreases usefulness. \# neighbors between 25 and 100 is often used.

\paragraph{Scoring Items from Neighborhoods:} We can use average scores or weighted average. 

\paragraph{Normalizing Data:} The reason doing normalization is because users rate differently (in scale). We can substract user mean rating or convert to z-score or substract item or item-user mean as we did in the main contents. 

\paragraph{Computing Similarities:} 1. Pearson correlation (Little data will make the similarity go to 1); 2. Spearman rank correlation; 3. Cosine similarity. 

\paragraph{Good Baseline Configuration:} Top N neighbors ($\sim$30); weighted averaging; user-mean or z-score normalization cosine similarity over normalized ratings.

\end{document}

